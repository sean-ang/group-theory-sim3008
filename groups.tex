\documentclass[12pt]{article}

\usepackage{math} %package containing various useful things for typsetting math stuff

%make nice headr and footer
\usepackage{lastpage}
\usepackage{fancyhdr}

\pagestyle{fancy}
\fancyhf{}
\lhead{\leftmark}
\rhead{\rightmark}
\cfoot{Page \thepage}

%make links and bookmarks in pdf
\usepackage{hyperref} 
\hypersetup{
	colorlinks		=false,
	linkcolor		=black,
	filecolor		=black,      
	urlcolor		=cyan,
	%bookmarks		=true,
}

\setlength{\parskip}{6pt}
\setlength{\parindent}{0pt}
\newcommand\ord{\text{ord}}
\newcommand\stab{\text{Stab}}
\newcommand\orb{\text{Orb}}

%note title
\title{Notes in Group Theory}
%feel free to add your name here! To be fair to everyone, we arrange the names in alphabetical order okay?
\author{Thum Sze Khai} 
\date{Last edited: \today}

\begin{document}

\begin{titlepage}
\thispagestyle{empty}
\maketitle

\vfill
Please send any error found to sze.khai0508(at)gmail.com.
\end{titlepage}

\newpage
\tableofcontents

\newpage

\section{Preliminaries}
We (usually) do not include proof for this section as these are learnt in Algebra I SIM2004.

\subsection{Groups}
	\begin{define}\label{definition-groups}
		(Definition of groups) Let $G$ be a set and $(\cdot)$ an operation on $G$. Suppose $G$ satisfies
		\begin{itemize}
			\item $\forall a,b \in G;\ a\cdot b\in G$ (closure),
			\item $\forall a,b,c,\in G;\ a\cdot(b\cdot c) = (a\cdot b) \cdot c$ (associativity),
			\item $\exists e\in G;\ a\cdot e=a=e\cdot a,\ \forall a\in G$ (identity),
			\item $\forall a\in G;\ \exists b\in G$ s.t. $a\cdot b=e=b\cdot a$, where we denote such $b = a\inv$ (inverses).
		\end{itemize}

		Then we say $(G,\cdot)$ is a group. 
	\end{define}

	We will omit the use of $(\cdot)$ whenever it's clear from context.

	\begin{theorem}\label{uniqueness-of-identity-inv}
		If $G$ is a group, then $G$ has a unique identity $e$. For every $a\in G$, $a\inv$ is also unique.
	\end{theorem}

	\begin{theorem}
		The Cayley table of a group $G$ is a latin square, but the converse is not necessarily true.
	\end{theorem}

\subsection{Subgroups}
	\begin{define}\label{definition-of-subgroup}
		(Subgroups) Let $G$ be a group and $H\subseteq G$. If $H$ satisfies
		\begin{itemize}
			\item $\forall a,b \in H;\ ab\in H$,
			\item $\forall a,b,c,\in H;\ a(bc) = (ab)c$,
			\item $\exists e\in H;\ ae=a=ea,\ \forall a\in H$,
			\item $\forall a\in H;\ \exists a\inv\in H$ s.t. $aa\inv=e=a\inv a$,
		\end{itemize}
		then we say $H$ is a subgroup of $G$ and write $H\leq G$.
	\end{define}

	\begin{theorem}
		Let $G$ be a group and $H\leq G$. If $e_H$ and $e_G$ are identity of $H$ and $G$ respectively, then $e_H=e_G$.
	\end{theorem}

	\begin{theorem}
		Let $G$ be a group, $H\leq G$, and $a\in H \subseteq G$. If $a\inv_H$ and $a\inv_G$ are inverses of $a$ in $H$ and $G$ respectively, then $a\inv_H=a\inv_G$.
	\end{theorem}

	\begin{theorem}
		(Subgroup criterion) Let $G$ be a group and $H \subseteq G$. Then $H$ is a subgroup iff $H\neq \emptyset$ and $xy\inv \in H$, $\forall x,y\in H$.
	\end{theorem}

\subsection{Lagrange theorem}
	\begin{define}
		(Coset) Let $G$ be a group and $H\leq G$. We define left coset $gH = \{gh : h\in H\}$ for $g\in G$. Similarly right coset is $Hg = \{hg : h\in H\}$
	\end{define}

	\begin{theorem}
		(Left coset equivalent criterion) Let $G$ be a group and $H\leq G$. For $a,b\in H$, $aH=bH$ iff $b\inv a \in H$.
	\end{theorem}

	\begin{theorem}
		Let $G$ be a group and $H\leq G$. If $a,b\in G$, then either $aH=bH$ or $aH \cap bH = \emptyset$.
	\end{theorem}

	\begin{theorem}
		Let $G$ be a group and $H\leq G$. If $a,b\in G$, then $\abs{aH}=\abs{bH}$. In particular $\abs{gH} = \abs{H}$ for any $g\in G$.
	\end{theorem}

	\begin{theorem}
		Let $G$ be a group and $H\leq G$. Then the set of cosets $\{gH : g\in G\}$ form a partition of $G$, i.e. $\cup\{gH : g\in G\} = G$ and the cosets are mutually disjoint.
	\end{theorem}

	\begin{theorem}\label{lagrange-theorem}
		(Lagrange) Let $G$ be a finite group and $H\leq G$. Then $\abs{H} \mid \abs{G}$.
	\end{theorem}

	\begin{define}
		Let $G$ be a group and $H\leq G$. Then the index of $H$ in $G$ is $[G:H] = \frac{\abs{G}}{\abs{H}}$.
	\end{define}

	\begin{corollary}
		Let $G$ be a group and $H\leq G$. Then $[G:H]$ is a positive integer.
	\end{corollary}

	\subsubsection{Corollaries of Lagrange theorem}

	\begin{theorem}
		Let $G$ be a group and $a\in G$. Then $\ord(a) \mid \abs{G}$.
	\end{theorem}

	\begin{corollary}\label{element-at-power-of-group-order-is-identity}
		Let $G$ be a group of order $n$ and $a\leq G$. Then $a^n = 1$.
	\end{corollary}

	\begin{theorem}
		If $G$ is a group of prime order $p$, then $G$ is cyclic.
	\end{theorem}
	\begin{proof}
		Let $a\in G$ be an non-identity element. Then $\abs{\angleb{a}} \mid \abs{G} = p$, hence $\angleb{a}$ has order $p$ since $a$ is non-identity. Therefore $\angleb{a} = G$ and $G$ is cyclic.
	\end{proof}

\subsection{Normal Subgroups}
	\begin{define}\label{definition-of-normal-subgroup}
		(Normal subgroup) Let $G$ be a group and $H\in G$. If $ghg\inv \in H$ for all $h\in H$, $g\in G$ then $H$ is a normal subgroup and we write $H \normsub G$.
	\end{define}

	\begin{theorem}
		(Normal subgroup criterion) Let $G$ be a group and $H\subseteq G$. The following are equivalent:
		\begin{enumerate}[(i)]
			\item $H\normsub G$
			\item $(H\neq \emptyset) \wedge (\forall x,y\in H;\ xy\inv \in H) \wedge (\forall h\in H, g\in G;\ ghg\inv \in H)$
			\item $H\leq G \wedge (\forall g\in G;\ gH=Hg)$
			\item $H\leq G \wedge (\forall g\in G;\ gHg\inv = H)$ 
		\end{enumerate}
	\end{theorem}

	\begin{theorem}
		Let $G$ be a group and $H\leq G$. If $[G:H] = 2$ then $H\normsub G$.
	\end{theorem}

	\begin{theorem}
		(Quotient group) Let $G$ be a group and $H\normsub G$. Let $G/H = \{gH : g\in G\}$ be the set of left cosets. Then $G/H$ forms a group with the operation
		$$(aH)(bH)=(ab)H,\ a,b\in H.$$
		We call $G/H$ the quotient group.
	\end{theorem}

	\begin{corollary}
		Let $G$ be a group and $H\normsub G$. Then $\abs{G/H} = [G:H]$.
	\end{corollary}

\subsection{Homomorphisms}
	\begin{define}
		(Homomorphism) Let $G$ and $H$ be groups. A function $f:G \to H$ is a homomorphism if $f(ab)=f(a)f(b)$ for all $a,b\in G$. If $f$ is surjective then we call $f$ an epimorphism. If $f$ is bijective we call $f$ an isomorphism.
	\end{define}

	\begin{define}
		(Kernel and image) Let $G$, $H$ be groups and $f: G\to H$ a homomorphism. Then the kernel of $f$ is
		$$\ker f = \{g\in G : f(g) = e_H\}$$
		and the image of $f$ is
		$$\text{Im} f = \{f(g) : g\in G\} = \{h : h\in H \text{ s.t. } \exists g\in G; f(g)=h\}$$
	\end{define}

	\begin{theorem}
		Let $G$ and $H$ be groups and $f: G\to H$ a homomorphism. Then $\ker f \leq G$ and $\text{Im} f \leq H$. Moreover, $\ker f \normsub G$.
	\end{theorem}

\subsection{Miscellaneous theorems}
	\begin{theorem}\label{subgroup-of-cyclic-group-is-cyclic}
		If $H$ is a subgroup of a cyclic group, then $H$ is cyclic.
	\end{theorem}
	\begin{proof}
		Let $H\leq G = \angleb{x}$. We may assume $\abs{H} > 1$. Then there exist $h\in H$ s.t. $h = x^i$ for some non-zero integer $i$. Yet if $i<0$, since $H$ is a group, then $h\inv = x^{-i} \in H$. Therefore there exist natural number $n$ where $x^n\in H$.

		Let $n_0$ be the smallest natural where $x^{n_0} \in H$. We claim that $H=\angleb{x^{n_0}}$. Clearly $\angleb{x^{n_0}} \subseteq H$. Let $y\in H$ where $y=x^i$ for some integer $i$. Similarly we may consider only $i>0$. By the division algorithm, there exist integer $q$ and $0\leq r < n$ where $i = qn_0 + r$. But
		$$y = x^i = x^{qn_0 + r} = x^r \in H.$$
		Thus we must have $r=0$ and $n_0 \mid i$. Hence $H \subseteq \angleb{x^{n_0}}$ and therefore $H=\angleb{x^{n_0}}$ as needed.
	\end{proof}

	\begin{theorem}
		Suppose $G$ is a cyclic group of order $n$. If $d\mid n$ then $G$ contains exactly one subgroup of order $d$.
	\end{theorem}
	\begin{proof}
		Let $G = \angleb{a}$, $H\leq G$ and $\abs{H} = d$. From theorem \ref{subgroup-of-cyclic-group-is-cyclic} we have $H = \angleb{a^i}$ for some integer $i$. By hypothesis (or Lagrange theorem) we have $n \mid d$. Furthermore, since $a^{id} = 1$ by corollary \ref{element-at-power-of-group-order-is-identity}, we have $id = kn$ for some integer $k$. Hence $i = k\left(\frac{n}{d}\right)$ and $a^i \in \angleb{a^{\frac{n}{d}}}$, therefore $H\subseteq \angleb{a^{\frac{n}{d}}}$.

		Note that $\ord(a^{\frac{n}{d}}) = d$ since for any natural $l$ where $a^{\frac{n}{d}l} = 1$ implies $\frac{n}{d}l = k'n$ for some integer $k'$, thus $d\mid l$ and $d\leq l$.

		Therefore $H = \angleb{a^{\frac{n}{d}}}$. Finally it is easy to see that for any $d\mid n$, $\angleb{a^{\frac{n}{d}}}\leq G$, and we're done.
	\end{proof}

\newpage
\section{Some important results}
This section contains results that I can't fit into anywhere.
\subsection{Isomorphism theorems}
	\begin{theorem}\label{first-isomorphism-theorem}
		(First isomorphism theorem) Let $f: G\to H$ be an epimorphism with kernel $K$. Then $K \normsub G$ and $G/K \cong H$.
	\end{theorem}
	\begin{proof}
		It is easy to show that $K \normsub G$. We define $h: G/K \to H$ where $h(gK) = f(g)$ for $g\in G$. 

		The function $h$ is well-defined since $g_1 K = g_2 K$ implies $g_2\inv g_1 \in K$ hence $f(g_2\inv g_1) = 1$ which implies $f(g_1)=f(g_2)$ as desired. Note that the reversed direction shows that $h$ is injective.

		Then, 
		$$h(g_1Kg_2K) = h(g_1g_2K) = f(g_1g_2) = f(g_1)f(g_2) = h(g_1K)h(g_2K)$$
		so $h$ is a homomorphism.

		Clearly $h$ is surjective. Therefore $h$ is an isomorphism, thus $G/K \cong H$.
	\end{proof}

	\begin{theorem}\label{second-isomorphism-theorem}
		(Second isomorphism theorem) Let $N$ and $T$ be subgroups of $G$ with $N \normsub G$. Then, $N\cap T \normsub T$ and $T/(N \cap T) \cong NT/N$.
	\end{theorem}

	Note on the quotient group $NT/N$: we have the following representation
	$$NT/N = \{ntN : nt\in NT\} = \{t(t\inv nt)N : nt\in NT\} = \{tN : t\in T\}$$
	which is much simpler

	\begin{proof}
		Let $a\in T$, $b\in N\cap T$. Clearly $aba\inv \in T$. Since $N\normsub G$, we have $aba\inv \in N$, hence $aba\inv \in N\cap T$ thus $N\cap T \normsub T$.

		Let $f: T \to NT/N$ where $f(a) = aN$ for $a\in T$. Clearly $f$ is a surjective homomorphism. Note that $f(a) = N$ iff $a\in N$, since $a\in T$, we have $a\in N\cap T$. Therefore $\ker f = N\cap T$. By theorem \ref{first-isomorphism-theorem} we have
		$$T/(N\cap T) \cong NT/T$$
		and we're done. 
	\end{proof}

	\begin{theorem}\label{third-isomorphism-theorem}
		(Third isomorphism theorem) Let $K\leq H\leq G$ with $K,H\normsub G$. Then $H/K \normsub G/K$ and 
		$$(G/K)/(H/K) \cong G/H.$$
	\end{theorem}

	\begin{proof}
		Let $g\in G,\ h\in H$. Then
		$$(gK)\inv (hK) (gK) = (g\inv h g)K \in H/K$$
		since $H$ is normal. Hence $H/K \normsub G/K$.

		Now let $g: G/K \to G/H$ by $g(aK) = aH$ for $a\in G$. Since $aK=bK \implies b\inv a \in K \leq H$ hence $b\inv a\in H$ thus $aH=bH$, therefore $g$ is well-defined.

		Clearly $g$ is a surjective homomorphism. Then $g(aK) = H$ iff $a\in H$, i.e. $aK \in H/K$. Thus $\ker g = H/K$. By theorem \ref{first-isomorphism-theorem}, we are done.
	\end{proof}

\subsection{Cartesian product of groups}
	\begin{define}
		Let $A,B$ be groups. We define the cartesian product of $A$ and $B$ as
		\begin{equation*}
			A\times B = \{(a,b) : a\in A, b\in B\}.
		\end{equation*}
	\end{define}

	\begin{theorem}
		Let $A,B$ be groups. Then $A\times B$ is a group under
		$$(a,b)*(c,d) = (ac,bd),\ \forall a,c\in A,\ b,d\in B.$$
	\end{theorem}
	\begin{proof}
		Let $(a,b),(c,d)\in A\times B$. Since $A,B$ are closed, 
		$$(a,b)*(c,d) = (ac,bd) \in A\times B$$

		Then, let $(e,f) \in A\times B$. We have
		\begin{multline*}
		[(a,b)*(c,d)]*(e,f) = (ac,bd)*(e,f) = ((ac)e, (bd)f)\\ 
		= (a(ce),b(df)) = (a,c) * (ce,df) = (a,c) * [(c,d) * (e,f)]$$
		\end{multline*}
		
		Note that $(1,1)\in A\times B$. Then for any $(a,b) \in A\times B$
		, $(a,b)*(1,1) = (a,b) = (1,1) * (a,b)$. Hence $(1,1)$ is the identity.

		If $(a,b)\in A\times B$, then $(a\inv,b\inv) \in A\times B$. Also,
		$$(a,b)*(a\inv,b\inv) = (1,1) = (a\inv,b\inv) * (a,b)$$
		and we're done.
	\end{proof}

	\begin{theorem}
		Let $A,B$ be groups and $G=A\times B$.Then $A\times \{1\}$ and $\{1\}\times B$ are normal subgroups of $G$.
	\end{theorem}
	\begin{proof}
		Let $(x,y) \in G$ and $a\in A$, hence $(a,1) \in A\times\{1\}$. Clearly, $A\times\{1\} \leq G$. Then
		$$(x,y)\inv * (a,1) * (x,y) = (x\inv ax,1) \in A\times\{1\}$$
		Hence $A\times\{1\}$ is normal. Similarly for $\{1\}\times B$.
	\end{proof}

	\begin{theorem}
		Let $A,B$ be groups, $H=A\times \{1\}$ and $K=\{1\}\times B$. Then, $H\cap K = \{1=(1,1)\}$, $H,K\normsub G$, and $G = HK$.
	\end{theorem}
	\begin{proof}
		(Basically trivial.)
	\end{proof}

	\begin{theorem}\label{direct-product-thm}
		(Internal direct product theorem) Let $H,K\normsub G$. If $H\cap K = \{1\}$ and $G=HK$, then $G\cong H\times K$.
	\end{theorem}
	\begin{proof}
		Let $f: G \to H\times K$\footnotemark where
		$$f(ab) = a\times b\ \forall a\in H, b\in K$$
		We shall show that $f$ is well-defined. Suppose we have $a_1, a_2\in H$, $b_1,b_2\in K$ such that $a_1b_1=a_2b_2$. Then $a_2\inv a_1 = b_2b_1\inv$, clearly the left hand side is in $H$, right hand side is in $K$, therefore they are identity since $H\cap K =\{1\}$, which yields $a_1=a_2$, $b_1=b_2$. Hence $f$ is well-defined.

		Now, $f(a_1b_1a_2b_2) = f(a_1 b_1 a_2 b_1\inv b_1 b_2)$. Since $H$ is normal, $a_1 b_1 a_2 b_1\inv \in H$, so
		$$f(a_1b_1a_2b_2) = (a_1 b_1 a_2 b_1\inv, b_1 b_2) = (a_1,b_1)(b_1 a_2 b_1\inv, b_2).$$
		Then, consider $b_1 a_2 b_1\inv a_2\inv$. On one hand, $b_1 (a_2 b_1 a_2\inv) \in K$, on the other hand $(b_1 a_2 b_1\inv) a_2\inv \in H$. Hence $b_1 a_2 b_1\inv a_2\inv = 1$ which implies $b_1 a_2 b_1\inv = a_2$.
		Therefore
		$$f(a_1b_1a_2b_2)= (a_1,b_1)(a_2,b_2) = f(a_1b_1)f(a_2b_2)$$
		and $f$ is a homomorphism.

		Clearly $f$ is surjective. For injectivity, if we have $f(a_1b_1)=f(a_2b_2)$ then $(a_1,b_1)=(a_2,b_2)$ which is equivalent to $a_1=a_2$ and $b_1=b_2$. Thus $f$ is bijective and $G\cong H\times K$ as desired.
	\end{proof}
	\footnotetext{(Unworthy comment from TSK) Actually we can take $f:H\times K \to G$. Then well-defined becomes trivial, the proof of homomorphism is still the same (more or less), surjectivity remains trivial, and the proof of injectivity is similar to proving well-defined in the original proof. So we only have to worry about two parts.}

	\begin{remark}
		Another formulation of theorem \ref{direct-product-thm} is $H,K\leq G$ where
		\begin{enumerate}[(i)]
			\item $H\cap K = \{1\},$
			\item for all $h\in H,k\in K$, we have $hk=kh$,
			\item $G=HK$.
		\end{enumerate}
		It is easy to see how (ii) and (iii) implies $H,K\normsub G$. If $H,K\normsub G$, then along with (i) we can show that $hkh\inv k\inv = 1$ hence $hk=kh$ for any $h\in H, k\in K$.
	\end{remark}

\newpage
\section{Symmetric groups}

\subsection{Introduction}
	\begin{define}
		Let $[n] = \{1,2,3,\dots,n\}$. A permutation on $[n]$ is a bijective function $\sigma: [n] \to [n]$.
	\end{define}

	\begin{theorem}
		The set of all permutation on $[n]$, $S_n$ with composition ($\circ$) forms a group. We call $S_n$ the symmetric group.
	\end{theorem}
	\begin{proof}
		Let $\sigma,\varphi\in S_n$. Since $\sigma$ and $\varphi$ are bijective, $\sigma \circ \varphi$ is also a bijective function on $[n]$, hence $\sigma \circ \varphi \in S_n$ for any $\sigma, \varphi \in S_n$.

		Now let $\psi \in S_n$. Since function composition is associative, we have $\sigma \circ (\varphi \circ \psi) = (\sigma \circ \varphi) \circ \psi$ for any $\sigma, \varphi, \psi \in S_n$.

		Let $1 : [n] \to [n]$ by $1(k) = k$ for $k\in [n]$. Note that $1\in S_n$. Then $\sigma \circ 1 (k) = \sigma(k) = 1\circ \sigma(k)$ for all $k\in [n]$.

		For any $\sigma \in S_n$, $\sigma\inv$ is also a bijective function, hence $\sigma\inv \in S_n$. Furthermore, $\sigma \circ \sigma\inv = 1 = \sigma\inv \circ \sigma$. Therefore $S_n$ is a group under composition as desired.
	\end{proof}

	\begin{remark}
		The symmetric group $S_n$ has order $n!$.
	\end{remark}

\subsection{Cycle notation}
	There are several ways to represent a permutation. A natural one is the \textbf{row notation}.

	Let $\sigma\in S_n$. We may write
	$$\sigma=
	\begin{pmatrix}
	1&2&3&\cdots&n\\
	\sigma(1)&\sigma(2)&\sigma(3)&\cdots&\sigma(n)
	\end{pmatrix}
	$$

	However, the row notation is slightly unwieldy. It takes up two lines and it hides some information from us. For example, it is not clear, from the notation, what is the order of $\sigma$. Also it's difficult to work out the composition.

	We present a new notation, the \textbf{cycle notation}. Consider the following permutation under $S_6$.

	$$
	\sigma=
	\begin{pmatrix}
	1&2&3&4&5&6\\
	2&3&1&5&4&6
	\end{pmatrix}
	$$

	Notice that the number $1,2,3$ forms a cycle of length 3. Similarly 4 and 5 forms a cycle of length 2. Consider the notation
	$$\sigma = (1\ 2\ 3)(4\ 5)$$

	We read the notation from right to left. Consider a more complicated cycle notation
	$$\psi = (1\ 3\ 4)(3\ 6)(2\ 4\ 6)(5\ 1)$$
	Following the notation from right to left, we have
	\begin{align*}
	&1\to 5\\
	&2\to 4\to 1\\
	&3\to 6\\
	&4\to 6\to 3\to 4\\
	&5\to 1\to 3\\
	&6\to 2
	\end{align*}
	which gives
	$$
	\psi=
	\begin{pmatrix}
	1&2&3&4&5&6\\
	5&1&6&4&3&2
	\end{pmatrix}
	$$
	So we can actually simplify $\psi$ to $\psi = (1\ 5\ 3\ 6\ 2)$.

	We say a cycle notation $\sigma = (\lambda_1^{(1)} \lambda_2^{(1)} \lambda_3^{(1)}\cdots \lambda_{l_1}^{(1)})(\lambda_1^{(2)} \lambda_2^{(2)} \lambda_3^{(2)}\cdots \lambda_{l_2}^{(2)})\cdots(\lambda_1^{(k)} \lambda_2^{(k)} \lambda_3^{(k)}\cdots \lambda_{l_k}^{(k)})$ is a \textbf{disjoint cycle notation} if each cycle are disjoint, i.e. no number appears in two cycle.

	\begin{theorem}
	Disjoint cycle commutes.
	\end{theorem}

	\begin{theorem}
	(Disjoint cycle notation works) Let $S_n$ be a symmetric group. Then all permutation in $S_n$ has a (essentially unique) disjoint cycle notation.
	\end{theorem}
	Essentially unique means that the order of the disjoint cycle doesn't matter, the ``rotation'' of individual cycle doesn't matter.
	
	\begin{proof}
	Let $\sigma\in S_n$. Consider the cycle $(1\ \sigma(1)\ \sigma^2(1)\ \sigma^3(1)\ \cdots)$. There must exist positive integers $k,l$ where $\sigma^k(1) = \sigma^l(1)$. Hence $\sigma^{k-l}(1) = 1$. Let $k$ be the smallest natural where $\sigma^k(1)=1$. Hence the first cycle is
	$$(1\ \sigma(1)\ \sigma^2(1)\ \sigma^3(1)\ \cdots\ \sigma^{k-1}(1)).$$
	Now pick $j = [n]\setminus \{1, \sigma(1), \sigma^2(1), \sigma^3(1), \cdots, \sigma^{k-1}(1)\}$. Similarly, we have a second cycle
	$$(j\ \sigma(j)\ \sigma^2(j)\ \cdots\ \sigma^l(j)).$$
	These cycle are disjoint, otherwise it is clear that they are the same cycle. We may repeat this until we exhausted all $1,2,\dots,n$.
	\end{proof}

	\textbf{Terminology:} We call a cycle of length $k$ a $k$-cycle. A 2-cycle is also called a transposition.

	\begin{theorem}
	Symmetric groups are generated by transposition.
	\end{theorem}
	\begin{proof}
	We only need to show that we may express any cycle as a product of transpositions. Consider a cycle $(a_1\ a_2\ a_3\ \cdots\ a_n)$. Note that
	$$(a_1\ a_n)(a_1\ a_{n-1})(a_1\ a_{n-2})\cdots(a_1\ a_3)(a_1\ a_2) = (a_1\ a_2\ a_3\ \cdots\ a_n)$$
	and we're done.
	\end{proof}

	\begin{remark}
	There is more than one way to express a cycle as product of transpositions. For example
	$$(a_1\ a_2\ a_3\ \cdots\ a_n) = (a_1\ a_2)(a_2\ a_3)(a_3\ a_4)\cdots(a_{n-1}\ a_n)$$
	\end{remark}

	\begin{theorem}
	A permutation with one and only one (non-singular) cycle of length $k$ has order $k$.
	\end{theorem}
	\begin{corollary}
	Let $\sigma\in S_n$, if $\sigma = c_1c_2c_3\cdots c_k$ where $c_i$ are disjoint cycles, and $c_i$ has length $l_i$, then $ord(\sigma)=\text{lcm}(l_1,l_2,\dots,l_k)$.
	\end{corollary}

\subsection{Sign of permutation}
	As noted, a permutation can be represented as different products of transposition.

	\begin{example}
		$$(1\ 2\ 3) = (1\ 2)(2\ 3) = (1\ 3)(1\ 2) = (1\ 2)(1\ 3)(1\ 2)(1\ 3) = (1\ 2)(1\ 3)(1\ 2)(1\ 3)(2\ 3)(1\ 3)(1\ 2)(1\ 3)$$
	\end{example}

	As we can see, neither the transpositions in the product nor the number of transposition are fixed. However, observe that the parity of the number of transposition remains the same for each product. 

	In fact, the parity of number of transposition is an invariant of the permutation. We may define the sign of permutation through this: if the product has even number of transposition then it is a even permutation; likewise for odd permutation.

	\begin{theorem}
		The sign of permutation is well-defined.
	\end{theorem}
	\begin{proof}
		We will show that the identity permutation $e$ is even. Let 
		$$e=\beta_1\beta_2\cdots \beta_m$$
		where $\beta_1,\beta_2,\dots,\beta_m$ are transposition. Suppose on the contrary that $m$ is odd. We may assume that this is the minimal example, i.e. $m$ is the smallest possible natural number where $e$ is a product of $m$ transposition.

		Now, if $m=1$ then we have an immediate contradiction. Therefore $m\geq 3$. Let $\beta_m = (a\ b)$. Now $b$ must appear within one of the transposition other than $\beta_m$, otherwise $b$ is swapped with $a$ but never returned, and thus the product is not identity. Let $r$ be the largest integer where $B_r$ contains $b$, and let $B_r = (b\ c)$. Consider that for $a,b,e,f$ distinct we have $(e\ f)(a\ b) = (a\ b)(e\ f)$ and $(a\ f)(a\ f) = (a\ b)(b\ f)$. So we may move $\beta_m$ down the product until we have
		$$e=\beta_1\beta_2\cdots\beta_r\beta_m\beta_{r+1}'\cdots\beta_{m-1}'$$

		Now if $c=a$ then we have $e=\beta_1\beta_2\cdots \beta_{r-1}\beta_{r+1}'\cdots \beta_{m-1}'$ violating the minimality of $m$. Otherwise $c\neq a$ and we have $(b\ c)(a\ b) = (a\ c)(b\ c)$.

		Note that now $a$ is not contained in $\beta_{r+1}'\cdots\beta_{m-1}'$. Therefore after the permutation \\$\beta_{r-1}(a\ c)\beta_r '\beta_{r+1}'\cdots \beta_{m-1}'$, $c$ takes the position of $a$, therefore similarly there must be a largest $s<r$ where $\beta_s$ contains $c$ and $\beta_s = (c\ d)$.

		Similarly we have
		$$e=\beta_1\beta_2\cdots\beta_s(a\ c)\beta_{s+1}'\cdots\beta_r'\beta_{r+1}'\cdots\beta_{m-1}'.$$ 
		If $a=d$ then we have a contradiction. Otherwise we may repeat the process again. But the process must end, i.e. we go back to the contradicting case or we reach the end and only one transposition contains $a$, another contradiction. Therefore $m$ cannot be odd and $e$ is even.

		Now, since $e$ is even, if we have two equal product of transpositions
		$$\beta_1\beta_2\cdots\beta_n = \beta_1'\beta_2'\cdots\beta_m'$$
		then
		$$e = \beta_m'\beta_{m-1}'\cdots\beta_1'\beta_1\beta_2\cdots\beta_n$$
		therefore $m+n$ is even, and $n$ and $m$ has the same parity, and we're done.
	\end{proof}

	\begin{theorem}
		The function $sgn:S_n \to \{1,-1\}$ such that for $\sigma \in S_n$ we have
		$$
		Sign(\sigma) = 
		\begin{cases}
			1 &: \sigma \text{ is even}\\
			-1 &: \sigma \text{ is odd}\\
		\end{cases}
		$$
		then $sgn$ is a surjective homomorphism.
	\end{theorem}

	\begin{define}
		The alternating group of order $n$ is $A_n = \ker sgn$.
	\end{define}

	\begin{theorem}
		$A_n \normsub S_n$
	\end{theorem}
	The proof is clear from the definition.

\newpage
\section{Group actions}
\subsection{Introduction}
	\begin{define}\label{group-action-definition}
		Let $G$ be a group and $X$ be a set. Let function $\theta : G\times X \to X$ satisfying $\theta(g,x) \in X$, $\theta(e,x) = x$ and $\theta(g_1,\theta(g_2,x)) = \theta(g_1g_2,x)$ for all $x\in X$, $g,g_1,g_2\in G$ and $e\in G$ is identity.

		We call $\theta$ a group action of $G$ on $X$.
	\end{define}

	We will drop the functional notation in favour of the ``left multiplication'' notation, i.e. $\theta(g,x) = gx$. But keep in mind that a group action does not have to be a left multiplication.

	\begin{define}
		Let $G$ acts on $X$ adn $x\in X$. We define the stabilizer of $x$ 
		$$\stab(x) = \{g\in G : gx=x\},$$
		and orbit of $x$
		$$\orb(x) = \{gx : g\in G\}.$$
	\end{define}

	Essentially, $\stab(x)$ are the identity actions on $x$, and $\orb(x)$ are the possible images of $x$ in the the action of $G$.

	\begin{theorem}
		Let $G$ acts on $X$ and $x\in X$. Then $\stab(x)\leq G$.
	\end{theorem}
	\begin{proof}
		Clearly $\stab(x) \subset G$ and $e\in \stab(x)$. Let $a,b\in \stab(x)$. Now $bx=x$ implies that $x=b\inv x$.\footnote{It is important to think of this not as left multiplication, but in the context of the third condition in definition \ref{group-action-definition}.} Then
		$$(ab\inv) x = a(b\inv x) = ax = x$$
		hence $ab\inv \in \stab(x)$ and $\stab(x)\leq G$ as desired.
	\end{proof}

	\begin{theorem}
		Let $G$ acts on $X$. Then the orbits of the action form a partition of $X$, i.e. for $x,y\in X$, either $\orb(x)=\orb(y)$ or $\orb(x)\cap\orb(y) = \emptyset$, and $X = \bigcup_{x\in X} \orb(x)$.
	\end{theorem}
	\begin{proof}
		Note that for all $x\in X$, $x=ex\in \orb(x)$. Hence $X = \bigcup_{x\in X} \orb(x)$.

		Let $x,y\in X$. Suppose $\orb(x)\cap\orb(y) \neq \emptyset$. Let $z\in \orb(x)\cap\orb(y)$. There exist $a,b\in G$ where
		$$z=ax=by.$$

		Let $w\in\orb(x)$, then $w=cx$ for some $c\in G$, and $w=c(a\inv z) = c(a\inv (by))\in \orb(y)$. Therefore $\orb(x)\subseteq\orb(y)$. Similarly $\orb(y) \subseteq \orb(x)$ and we're done.
	\end{proof}

	\begin{corollary}
		There exist $x_1,x_2\dots,x_m\in X$ such that $\orb(x_1),\orb(x_2),\dots,\orb(x_m)$ forms a partition of $X$. Also
		$$\abs{X} = \bigcup_{i=1}^m \abs{\orb(x_i)}$$
	\end{corollary}

	\begin{theorem}(Orbit-Stabilizer theorem)
		Let $G$ acts on $X$ and $x\in X$. Then
		$$\abs{\orb(x)} = \abs{G:\stab(x)}$$
		or in other words
		$$\abs{G} = \abs{\orb(x)}\abs{\stab(x)}$$
	\end{theorem}
	\begin{proof}
		Let $f: G/\stab(x) \to \orb(x)$ by $f(a\stab(x)) = ax$ for $a\in G$. Since $a\stab(x) = b\stab(x)$ iff $b\inv a\in \stab(x)$ iff $b\inv ax=x$ iff $bx=ax$ so $f$ is well-defined and one-to-one. Clearly $f$ is onto, and thus $f$ is a bijective. Therefore
		$$\abs{G:\stab(x)} = \abs{G/\stab(x)} = \abs{\orb(x)}$$
		and we're done.
	\end{proof}

	Now we present an alternate proof to theorem \ref{lagrange-theorem}. Let $H\leq G$, and let $H$ acts on $G$ by left multiplication. (TSK: Work in progress)

\subsection{Conjugacy}
	

\newpage
\section{Sylow p-subgroups}




\newpage
\appendix

\section{Recommended references}
	\begin{enumerate}
	\item Hall, M., The theory of Groups. Dover Publications; Reprint edition, New York, 2018.
	\item Barnard, T., Neil, H., Discovering Group Theory: A Transition to Advanced Mathematics, Taylor \& Francis Ltd, London, 2016.
	\item Rotman, J.J., An introduction to the theory of groups, 4th edition. Springer-Verlag, New York, 1999.
	\end{enumerate}

\end{document}